\chapter{Introduction}
\label{chap:introduction}

Rise of data: KITTI (2012, http://www.cvlibs.net/datasets/kitti/)
Rise of ML: AlexNet (2011/2012 https://en.wikipedia.org/wiki/AlexNet)
Rise of autonomous driving: DARPA (2004,5,7 https://en.wikipedia.org/wiki/DARPA_Grand_Challenge)

Lots of prototyping, now (2021) the issue of scaling autonomous vehicles in the real world is coming. Corner cases get more and more important, since the vehicles are very mature in well defined Operational Design Domains (ODDs).

\subsection*{Taxonomy Autonomous Vehicles}

\begin{enumerate}
    \item \cite{sae_j3016c_2021}
    \item \cite{bundesregierung_entwurf_2021}
\end{enumerate}

\section{Scope and Assumptions}

\begin{itemize}
    \item Natural, external corner cases, no attacks or hardware-defects
\item Urban Scenarios (no highspeed)
\item Focus on simulation (CARLA) due the issues that ccs are often impossible or dangerous to perform in real life
\item Multi-modal sensor setup from a typical AD stack (camera, lidar, radar)
\item Encapsulated methods, no error propagation (perfect eprception in planning chapter)
\item Research on functions, no validation/verification

Debate about why finding cc and re-training on them is non-sense (my idea: generalize for any kind of cc).


\end{itemize}

\section{Einordnung und Wissenschaftlicher Beitrag}

\section{Aufbau der Arbeit}


