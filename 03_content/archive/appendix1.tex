\chapter{Appendix 1}
\label{chap:appendix1}


\section{Partikelschwarmoptimierung}
\label{appendix:pso}
Die Partikelschwarmoptimierung ist ein biologisch motiviertes Optimierungsverfahren, das 1995 durch Kennedy et al. vorgestellt \cite{Kennedy1995}.
Es bildet das Verhalten eines Schwarmes nach, dessen Individuen bzw. Partikel $n$ einerseits eigenständig nach einem Optimum suchen, und andererseits durch das Verhalten des erfolgreichsten Schwarm-Individuums $b$ beeinflusst werden.
Dabei wandert jedes Partikel mit einer sich verändernden Geschwindigkeit $\vec{v}_n$ durch den Suchraum $X$ des Optimierungsproblems und speichert dabei seine bisher beste erreichte Position $\vec{x}_{n\,\text{best}}$.
Da der Schwarm ein randomisiert exploratives Verhalten aufweist, können lokale Minima bei passender Parametrierung vermieden werden.
In dieser Arbeit wird die PSO dafür genutzt, nichtlineare Optimierungsprobleme anzugehen.
Dafür gilt:

Maximiere die Bewertungsfunktion $f(\vec{x})$ unter der Nebenbedingung $\vec{x}\in X$ mit $f\colon W\to \mathbb{R}$ eine reellwertige Funktion und $X \subseteq W$. 
Die zulässige Menge $X$ ist durch ihren konkreten Wertebereich beschrieben.

Die Implementierung iteriert nach einer Initialisierung durch drei Phasen, bis ein Maximum an Iterationen erreicht ist, oder ein Abbruchkriterium für das Ergebnis von $f(\vec{x})$ erreicht wurde.

Zunächst wird ein Schwarm mit einer festen Anzahl $N$ an Partikeln erzeugt, indem jedem Individuum $n \in N$ ein randomisierter Startwert $\vec{x}_n$ zugewiesen wird.
Danach startet die Optimierung:
\begin{enumerate}
	\item Bewerte: Berechne $f(\vec{x}_n)$ für alle $n$ und setze $\vec{x}_{n\,\text{best}}$, falls $f(\vec{x}_n) > \vec{x}_{n\,\text{best}}$
	\item Ermittle Schwarm-Besten $\vec{x}_b = \argmax_{n \in N} (f(\vec{x}_n))$ aus allen $N$ Partikeln
	\item Berechne neue Partikel-Geschwindigkeit: \\
	$\vec{v}_n = \vec{v}_n 
	+ \big( \alpha \cdot p_1 \cdot (\vec{x}_{n\,\text{best}} -  \vec{x}_n ) \big) 
	+ \big( \beta \cdot p_2 \cdot (\vec{x}_b - \vec{x}_n) \big)$ \\
	mit Zufallsvariablen $p_1, p_2 \in [0, 1]$
	\item Verschiebe Partikel: $\vec{x}_n = \vec{x}_n \cdot \vec{v}_n$
\end{enumerate}
Dabei kann über die Parameter $\alpha$ und $\beta$ gesteuert werden, wie stark Partikel von ihrem eigenen Optimum $\vec{x}_{n\,\text{best}}$ oder dem Schwarm Optimum $\vec{b}$ angezogen werden.
Durch die Zufallsvariablen $p_1$ und $p_2$ streut der Schwarm und konvergiert nicht direkt gegen ein einziges Maximum.
Analog kann über entsprechende Umstellungen $\text{max} \to \text{min}$ auch ein Minimierungsproblem gelöst werden.


